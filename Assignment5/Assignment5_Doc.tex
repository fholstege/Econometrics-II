% Options for packages loaded elsewhere
\PassOptionsToPackage{unicode}{hyperref}
\PassOptionsToPackage{hyphens}{url}
%
\documentclass[
]{article}
\usepackage{lmodern}
\usepackage{amssymb,amsmath}
\usepackage{ifxetex,ifluatex}
\ifnum 0\ifxetex 1\fi\ifluatex 1\fi=0 % if pdftex
  \usepackage[T1]{fontenc}
  \usepackage[utf8]{inputenc}
  \usepackage{textcomp} % provide euro and other symbols
\else % if luatex or xetex
  \usepackage{unicode-math}
  \defaultfontfeatures{Scale=MatchLowercase}
  \defaultfontfeatures[\rmfamily]{Ligatures=TeX,Scale=1}
\fi
% Use upquote if available, for straight quotes in verbatim environments
\IfFileExists{upquote.sty}{\usepackage{upquote}}{}
\IfFileExists{microtype.sty}{% use microtype if available
  \usepackage[]{microtype}
  \UseMicrotypeSet[protrusion]{basicmath} % disable protrusion for tt fonts
}{}
\makeatletter
\@ifundefined{KOMAClassName}{% if non-KOMA class
  \IfFileExists{parskip.sty}{%
    \usepackage{parskip}
  }{% else
    \setlength{\parindent}{0pt}
    \setlength{\parskip}{6pt plus 2pt minus 1pt}}
}{% if KOMA class
  \KOMAoptions{parskip=half}}
\makeatother
\usepackage{xcolor}
\IfFileExists{xurl.sty}{\usepackage{xurl}}{} % add URL line breaks if available
\IfFileExists{bookmark.sty}{\usepackage{bookmark}}{\usepackage{hyperref}}
\hypersetup{
  pdftitle={Assignment 5 - Econometrics II},
  pdfauthor={Floris Holstege, Stanislav Avdeev},
  hidelinks,
  pdfcreator={LaTeX via pandoc}}
\urlstyle{same} % disable monospaced font for URLs
\usepackage[margin=1in]{geometry}
\usepackage{graphicx,grffile}
\makeatletter
\def\maxwidth{\ifdim\Gin@nat@width>\linewidth\linewidth\else\Gin@nat@width\fi}
\def\maxheight{\ifdim\Gin@nat@height>\textheight\textheight\else\Gin@nat@height\fi}
\makeatother
% Scale images if necessary, so that they will not overflow the page
% margins by default, and it is still possible to overwrite the defaults
% using explicit options in \includegraphics[width, height, ...]{}
\setkeys{Gin}{width=\maxwidth,height=\maxheight,keepaspectratio}
% Set default figure placement to htbp
\makeatletter
\def\fps@figure{htbp}
\makeatother
\setlength{\emergencystretch}{3em} % prevent overfull lines
\providecommand{\tightlist}{%
  \setlength{\itemsep}{0pt}\setlength{\parskip}{0pt}}
\setcounter{secnumdepth}{-\maxdimen} % remove section numbering

\title{Assignment 5 - Econometrics II}
\author{Floris Holstege, Stanislav Avdeev}
\date{}

\begin{document}
\maketitle

\hypertarget{section}{%
\subsection{1}\label{section}}

This approach is a difference-in-difference approach, which can be
written in OLS form as:

\begin{align*}
  y_{g1} - y_{g0} &= (a_1 - a_0) + \delta D_g + (U_{g1} - U_{g0}) \\
  &= B_0 + \delta D_g + U_g
\end{align*}

Where \(y_{g1}, y_{g0}\) is the dependent variable (grade point average)
at after the first year (\(t = 0\)), and after the second year
(\(t = 1\)). \(B_0\) is the trend and \(\delta\) is the effect of
treatment \(D_g\), in this case providing housing if one has a GPA
higher than 8. \(U_g\) are the errrors of this model.

We know that OLS is consistent as long as \(D_g\) and \(U_g\) are
uncorrelated. In order for this to hold, the treatment needs to be
independent of the outcome variable. That is not the case here, as the
outcome variable consists partially of first year grades (\(y_{g0}\)),
which also determine whether or not someone receives the treatment.
Hence, we would not expect this approach to give a consistent estimator.

\hypertarget{section-1}{%
\subsection{2}\label{section-1}}

\hypertarget{i}{%
\subsubsection{I)}\label{i}}

A bivariate model of sex ratio and \% out-of-wedlock births, using only
observations from the year before the war, finds no statistically
significant effect from sex ratio on \% out-of-wedlock births (robust
standard errors), suggesting that position of men in the marriage market
does not improve with a reduction in de sex ratio.

\begin{table}[!htbp] \centering 
  \caption{} 
  \label{} 
\begin{tabular}{@{\extracolsep{5pt}}lc} 
\\[-1.8ex]\hline 
\hline \\[-1.8ex] 
 & \multicolumn{1}{c}{\textit{Dependent variable:}} \\ 
\cline{2-2} 
\\[-1.8ex] & \% of out-of-wedlock births \\ 
\hline \\[-1.8ex] 
 Sex Ratio & $-$0.089 \\ 
  & (5.050) \\ 
  & \\ 
 Constant & 6.772 \\ 
  & (5.786) \\ 
  & \\ 
\hline \\[-1.8ex] 
Observations & 87 \\ 
Adjusted R$^{2}$ & $-$0.012 \\ 
\hline 
\hline \\[-1.8ex] 
\textit{Note:}  & \multicolumn{1}{r}{$^{*}$p$<$0.1; $^{**}$p$<$0.05; $^{***}$p$<$0.01} \\ 
\end{tabular} 
\end{table}

However, there are some obvious problems with this approach. The problem
with this approach is that there might be department-specific attributes
that could mean the the \% of births out of wedlock are structually
higher or lower. Using OLS, while ignoring these department-specific
effects results in biased estimations.

We can improve upon this approach by using a difference-in-difference
approach, with the military mortality rate as a treatment. In the French
army, the vast majority of armed forces were men. A high military
mortality rate should thus decrease the ratio of men to women,
increasing the position of men in the wedding market after the war,
subsequently leading (if you believe the argument made in the exercise)
to more out-of-wedlock births.

If we use difference-in-difference, we avoid the previously mentioned
problem, if one believes that the department-specific attributes are
consistent over the two periods.

\hypertarget{ii}{%
\subsubsection{II)}\label{ii}}

\begin{table}[]
\begin{tabular}{r|r|r|r}
                  & Before the war & After the war & Difference \\
                  \hline
 Above the median&  5.09 & 6.15 & 1.06    \\
 \hline
 Below the median&  7.96 & 8.45 & 0.49   \\
 \hline 
 Difference-in-Difference & & & 0.57 \\
\end{tabular}
\end{table}

\hypertarget{iii}{%
\subsubsection{III)}\label{iii}}

We run the following model

\begin{align*}
  \% of out-of-wedlock births = \beta_0 + \beta_1 (Post \cdot \text{Military mortality}) + \beta_2 Post + U
\end{align*}

In which the dependent variable is the \% of out-of-wedlock births,
``Post'' is a dummy variable, with value 1 if this dependent variable
was measured after the war, and 0 if before.

Before we make any conclusions, how should we interpret the coefficients
\(\beta_1, \beta_2\)?. For \(\beta_1\), the interaction term is simply
the effect of military mortality on \% of out-of-wedlock births, if this
\% was measured after the war. This is appropriate, because if the \%
was measured before the war, we would not yet know the potential effect
of the military mortality rate. For \(\beta_2\), this simply measures
the (average) difference between \% of out-of-wedlock births before and
after the war. This is important, because it ensures that any effect
attributed to the `Military mortality' variable is not due to a general
increase in out-of-wedlock births after the war.

Table \ref{fig:mortality_results} shows the results for the
abovementioned model, with robust standard erros, and with and without
department dummies. When we do not apply department dummies, the model
shows that the `post' variable is statistically significant, indicating
that in general the \% of out-of-wedlock births increased by 9.4\% after
the war. But the interaction variable between `post' and military
mortality is also statistically significant, indicating that one \%
increase in the mortality rate on average led to an 0.5\% decrease in
the out-of-wedlock births, suggesting that the position of males in the
wedding market only worsened as the sex ratio decreased.

\begin{table}[!htbp] \centering 
  \caption{} 
  \label{fig:mortality_results} 
\begin{tabular}{@{\extracolsep{5pt}}lcc} 
\\[-1.8ex]\hline 
\hline \\[-1.8ex] 
 & \multicolumn{2}{c}{\textit{Dependent variable:}} \\ 
\cline{2-3} 
\\[-1.8ex] & \multicolumn{2}{c}{illeg} \\ 
\\[-1.8ex] & (No department dummies) & (With department dummies)\\ 
\hline \\[-1.8ex] 
 post\_mortality & $-$0.515$^{***}$ & 0.148$^{***}$ \\ 
  & (0.151) & (0.049) \\ 
  & & \\ 
 post & 9.376$^{***}$ & $-$1.740$^{**}$ \\ 
  & (2.689) & (0.872) \\ 
  & & \\ 
 Constant & 6.672$^{***}$ & 6.360$^{***}$ \\ 
  & (0.380) & (0.071) \\ 
  & & \\ 
\hline \\[-1.8ex] 
Observations & 174 & 174 \\ 
Adjusted R$^{2}$ & 0.085 & 0.958 \\ 
\hline 
\hline \\[-1.8ex] 
\textit{Note:}  & \multicolumn{2}{r}{$^{*}$p$<$0.1; $^{**}$p$<$0.05; $^{***}$p$<$0.01} \\ 
\end{tabular} 
\end{table}

\hypertarget{iv}{%
\subsubsection{IV)}\label{iv}}

Once we include dummies for each department, the results are reversed:
the `post' variable is still statistically significant, but now
indicates that the \% of out-of-wedlock births decreased by 1.7\% after
the war, and the interaction variable between `post' and military
mortality indicates that one \% increase in the mortality rate on
average led to an 0.15\% increase in the out-of-wedlock birth,
suggesting that the position of males in the wedding market only
worsened as the sex ratio increased.

How is this possible? When we use a dummy for each department, we assume
that there is a fixed effect (both before and after the war) per
department that affects the the \% of out-of-wedlock births. We prefer
this specification, because it seems plausible that certain aspects of
department, such as socio-economic circumstances and social and cultural
norms in particular areas, should affect the rate of out-of-wedlock
births.

\hypertarget{v}{%
\subsubsection{V)}\label{v}}

The key assumption we make is the so-called `common trend' assumption,
which states that the difference between the control and treatment
(existent because of the different make-up of the control and treatment
group) remains constant over time. In our case, we assume that the trend
in the difference in out of wedlock births between departments that
ended up having with high and low military mortality rates continued.

How can we test if this assumption is reasonable? One of the easiest and
most intuitive ways is to simply plot the difference in out of wedlock
births between departments over time, and see if the difference is
constant or rather time-varying. But we cannot use this test in this
case, since we only have data for two moments (before and after the
war), instead of several years leading up to the war.

\end{document}
