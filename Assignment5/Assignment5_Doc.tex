% Options for packages loaded elsewhere
\PassOptionsToPackage{unicode}{hyperref}
\PassOptionsToPackage{hyphens}{url}
%
\documentclass[
]{article}
\usepackage{lmodern}
\usepackage{amssymb,amsmath}
\usepackage{ifxetex,ifluatex}
\ifnum 0\ifxetex 1\fi\ifluatex 1\fi=0 % if pdftex
  \usepackage[T1]{fontenc}
  \usepackage[utf8]{inputenc}
  \usepackage{textcomp} % provide euro and other symbols
\else % if luatex or xetex
  \usepackage{unicode-math}
  \defaultfontfeatures{Scale=MatchLowercase}
  \defaultfontfeatures[\rmfamily]{Ligatures=TeX,Scale=1}
\fi
% Use upquote if available, for straight quotes in verbatim environments
\IfFileExists{upquote.sty}{\usepackage{upquote}}{}
\IfFileExists{microtype.sty}{% use microtype if available
  \usepackage[]{microtype}
  \UseMicrotypeSet[protrusion]{basicmath} % disable protrusion for tt fonts
}{}
\makeatletter
\@ifundefined{KOMAClassName}{% if non-KOMA class
  \IfFileExists{parskip.sty}{%
    \usepackage{parskip}
  }{% else
    \setlength{\parindent}{0pt}
    \setlength{\parskip}{6pt plus 2pt minus 1pt}}
}{% if KOMA class
  \KOMAoptions{parskip=half}}
\makeatother
\usepackage{xcolor}
\IfFileExists{xurl.sty}{\usepackage{xurl}}{} % add URL line breaks if available
\IfFileExists{bookmark.sty}{\usepackage{bookmark}}{\usepackage{hyperref}}
\hypersetup{
  pdftitle={Assignment 5 - Econometrics II},
  pdfauthor={Floris Holstege, Stanislav Avdeev},
  hidelinks,
  pdfcreator={LaTeX via pandoc}}
\urlstyle{same} % disable monospaced font for URLs
\usepackage[margin=1in]{geometry}
\usepackage{graphicx,grffile}
\makeatletter
\def\maxwidth{\ifdim\Gin@nat@width>\linewidth\linewidth\else\Gin@nat@width\fi}
\def\maxheight{\ifdim\Gin@nat@height>\textheight\textheight\else\Gin@nat@height\fi}
\makeatother
% Scale images if necessary, so that they will not overflow the page
% margins by default, and it is still possible to overwrite the defaults
% using explicit options in \includegraphics[width, height, ...]{}
\setkeys{Gin}{width=\maxwidth,height=\maxheight,keepaspectratio}
% Set default figure placement to htbp
\makeatletter
\def\fps@figure{htbp}
\makeatother
\setlength{\emergencystretch}{3em} % prevent overfull lines
\providecommand{\tightlist}{%
  \setlength{\itemsep}{0pt}\setlength{\parskip}{0pt}}
\setcounter{secnumdepth}{-\maxdimen} % remove section numbering

\title{Assignment 5 - Econometrics II}
\author{Floris Holstege, Stanislav Avdeev}
\date{}

\begin{document}
\maketitle

\hypertarget{section}{%
\subsection{1}\label{section}}

This approach is a difference-in-difference approach, which can be
written in OLS form as:

\begin{align*}
  y_{g1} - y_{g0} &= (a_1 - a_0) + \delta D_g + (U_{g1} - U_{g0}) \\
                  &= B_0 + \delta D_g + U_g
\end{align*}

Where \(y_{g1}, y_{g0}\) is the dependent variable (grade point average)
at after the first year (\(t = 0\)), and after the second year
(\(t = 1\)). \(B_0\) is the trend and \(\delta\) is the effect of
treatment \(D_g\), in this case providing housing if one has a GPA
higher than 8. \(U_g\) are the errrors of this model.

We know that OLS is consistent as long as \(D_g\) and \(U_g\) are
uncorrelated. In order for this to hold, the treatment needs to be
independent of the outcome variable. That is not the case here, as the
outcome variable consists partially of first year grades (\(y_{g0}\)),
which also determine whether or not someone receives the treatment.
Hence, we would not expect this approach to give a consistent estimator.

\hypertarget{section-1}{%
\subsection{2}\label{section-1}}

A bivariate model of sex ratio and \% out-of-wedlock births, using only
observations from the year before the war, finds no statistically
significant effect from sex ratio on \% out-of-wedlock births (robust
standard errors), suggesting that position of men in the marriage market
does not improve with a reduction in de sex ratio.

\begin{table}[!htbp] \centering 
  \caption{} 
  \label{} 
\begin{tabular}{@{\extracolsep{5pt}}lc} 
\\[-1.8ex]\hline 
\hline \\[-1.8ex] 
 & \multicolumn{1}{c}{\textit{Dependent variable:}} \\ 
\cline{2-2} 
\\[-1.8ex] & \% of out-of-wedlock births \\ 
\hline \\[-1.8ex] 
 Sex Ratio & $-$0.089 \\ 
  & (5.050) \\ 
  & \\ 
 Constant & 6.772 \\ 
  & (5.786) \\ 
  & \\ 
\hline \\[-1.8ex] 
Observations & 87 \\ 
Adjusted R$^{2}$ & $-$0.012 \\ 
\hline 
\hline \\[-1.8ex] 
\textit{Note:}  & \multicolumn{1}{r}{$^{*}$p$<$0.1; $^{**}$p$<$0.05; $^{***}$p$<$0.01} \\ 
\end{tabular} 
\end{table}

However, there are some obvious problems with this approach. The problem
with this approach is that there might be department-specific attributes
that could mean the the \% of births out of wedlock are structually
higher or lower. Using OLS, while ignoring these department-specific
effects results in biased estimations.

We can improve upon this approach by using a difference-in-difference
approach, with the military mortality rate as a treatment. In the French
army, the vast majority of armed forces were men. A high military
mortality rate should thus decrease the ratio of men to women,
increasing the position of men in the market after the war, subsequently
leading (if you believe the argument made in the exercise) to more
out-of-wedlock births.

If we use difference-in-difference, we avoid the previously mentioned
problem, if one believes that the department-specific attributes are
consistent over the two periods.

\hypertarget{section-2}{%
\subsection{3}\label{section-2}}

\begin{table}[]
\begin{tabular}{rrrr}
                  & Before the war & After the war & Difference
 Above the median&  5.09 & 6.15 & 1.06    \\
 Below the median&  7.96 & 8.45 & 0.49   \\
 \vline \\
 Difference-in-Difference & & & 0.57 
\end{tabular}
\end{table}

\end{document}
